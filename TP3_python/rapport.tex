\documentclass{article}
\usepackage[utf8]{inputenc}
\usepackage{graphicx}
\usepackage{amsmath}
\usepackage{amsfonts}
\usepackage{amsbsy}
\usepackage{mathrsfs}
\usepackage{appendix}
\usepackage{amsthm}
\usepackage{bbold}
\usepackage{epstopdf}
\usepackage{stmaryrd}
\usepackage{grffile}
\usepackage{subcaption}
\newcommand{\bs}[1]{\boldsymbol{#1}}
\newcommand{\bb}[1]{\mathbf{#1}}
\newcommand{\pd}[2]{\frac{\partial {#1}}{\partial {#2}}}
\newcommand{\parti}[1]{\pd{}{#1}}
\newcommand{\bigCI}{\mathrel{\text{\scalebox{1.07}{$\perp\mkern-10mu\perp$}}}}
\title{TP3 for Reinforcement Learning}
\author{Leman FENG\\ Email: flm8620@gmail.com\\Website: lemanfeng.com}

\begin{document}
	\maketitle
	\section*{Q1 \& Q2}
	I choose by default $N=100, T=100$, and initialize with $\theta = -0.1$.
	
	For the constant step gradient descent, step = $0.001$ diverges, step=$0.0001$ converges(Figure \ref{fig:constant}).
	
	For the Annealing Step. I defined the schema as step = $\frac{\alpha}{t+10}$ where $\alpha$ is the learning rate and $t$ is the iteration. Compare to Constant step schema, Annealing step can tolerate a large learning rate, but it converges slowly (Figure \ref{fig:annealing}). We can see with $N=100, \alpha=0.01$, the performance is good.
	
	For Adam Step, this algorithm is more robust. I tried with $\alpha=0.01$ and $N=100, n_iter=100$, and again with $N=10, n_iter=1000$. Both tests converged (Figure \ref{fig:adam}). $N=10$ will lead to a high variance of gradient. With same step = $0.01$, Annealing stepper will diverge so fast, but Adam stepper is still stable. So we can say Adam stepper is suitable for stochastic gradient method.
	
	To answer to Question 1, small $N$ will lead to large variance, so small $N$ should be used with small step $\alpha$. As shown in Figure \ref{fig:annealing}, with $N=100, \alpha=0.01$, annealing stepper converges. But then I changed $N$ to $20$, it diverged. Again with a small step $\alpha=0.0005$, Annealing stepper can converge with $N=20$
	

	
	
	\begin{figure}
		\centering
		\begin{subfigure}[b]{0.49\textwidth}
			\centering
			\includegraphics[width=\textwidth]{Avg_return_N=100_T=100_ConstantStep=0.001.eps}
			\caption{Avg Return, Constant Step 0.001}
		\end{subfigure}
		\begin{subfigure}[b]{0.49\textwidth}
			\centering
			\includegraphics[width=\textwidth]{Theta_N=100_T=100_ConstantStep=0.001.eps}
			\caption{Theta, Constant Step 0.001}
		\end{subfigure}
	
		\begin{subfigure}[b]{0.49\textwidth}
			\centering
			\includegraphics[width=\textwidth]{Avg_return_N=100_T=100_ConstantStep=0.0001.eps}
			\caption{Avg Return, Constant Step 0.0001}
		\end{subfigure}
		\begin{subfigure}[b]{0.49\textwidth}
			\centering
			\includegraphics[width=\textwidth]{Theta_N=100_T=100_ConstantStep=0.0001.eps}
			\caption{Theta, Constant Step 0.0001}
		\end{subfigure}
	\caption{Constant Step}
	\label{fig:constant}
	\end{figure}

\begin{figure}
	\centering
	\begin{subfigure}[b]{0.49\textwidth}
		\centering
		\includegraphics[width=\textwidth]{Avg_return_N=100_T=100_AnnealingStep=0.001.eps}
		\caption{Avg Return, Annealing Step 0.001}
	\end{subfigure}
	\begin{subfigure}[b]{0.49\textwidth}
		\centering
		\includegraphics[width=\textwidth]{Theta_N=100_T=100_AnnealingStep=0.001.eps}
		\caption{Theta, Annealing Step 0.001}
	\end{subfigure}

	\begin{subfigure}[b]{0.49\textwidth}
		\centering
		\includegraphics[width=\textwidth]{Avg_return_N=100_T=100_AnnealingStep=0.01.eps}
		\caption{Avg Return, Annealing Step 0.01}
	\end{subfigure}
	\begin{subfigure}[b]{0.49\textwidth}
		\centering
		\includegraphics[width=\textwidth]{Theta_N=100_T=100_AnnealingStep=0.01.eps}
		\caption{Theta, Annealing Step 0.01}
	\end{subfigure}

	\begin{subfigure}[b]{0.49\textwidth}
		\centering
		\includegraphics[width=\textwidth]{Avg_return_N=20_T=100_AnnealingStep=0.0005.eps}
		\caption{Avg Return, Annealing Step 0.0005}
	\end{subfigure}
	\begin{subfigure}[b]{0.49\textwidth}
		\centering
		\includegraphics[width=\textwidth]{Theta_N=20_T=100_AnnealingStep=0.0005.eps}
		\caption{Theta, Annealing Step 0.0005}
	\end{subfigure}
	\caption{Annealing Step}
	\label{fig:annealing}
\end{figure}


\begin{figure}
	\centering
	\begin{subfigure}[b]{0.49\textwidth}
		\centering
		\includegraphics[width=\textwidth]{Avg_return_N=100_T=100_AdamStep=0.01.eps}
		\caption{Avg Return, Adam Step N=100}
	\end{subfigure}
	\begin{subfigure}[b]{0.49\textwidth}
		\centering
		\includegraphics[width=\textwidth]{Theta_N=100_T=100_AdamStep=0.01.eps}
		\caption{Theta, Adam Step N=100}
	\end{subfigure}
	
	\begin{subfigure}[b]{0.49\textwidth}
		\centering
		\includegraphics[width=\textwidth]{Avg_return_N=10_T=100_AdamStep=0.01.eps}
		\caption{Avg Return, Adam Step N=10}
	\end{subfigure}
	\begin{subfigure}[b]{0.49\textwidth}
		\centering
		\includegraphics[width=\textwidth]{Theta_N=10_T=100_AdamStep=0.01.eps}
		\caption{Theta, Adam Step N=10}
	\end{subfigure}
	\caption{Adam Step}
	\label{fig:adam}
\end{figure}

	\section*{Q3}
	To have some generality, I changed $\phi$ to $[1,s,a,s^2, a^2, sa]$. FQI still works very well. It converges to theoretical optimum in 5 iterations(Figure \ref{fig:fqi}). $J(\pi_k)$ is shown in Figure \ref{fig:J}
	
	\section*{Cart-pole}
	I found the cart-pole code, so I applied Policy gradient method to cart-pole. Action space is $\{0,1\}$, state space is $\mathbb{R}^4$. I use Gibbs model and set 
	$$
	Q_\theta(s,a=0)= \theta_0^T s, Q_\theta(s,a=1)= \theta_1^T s, 
	$$
	Because we want cart-pole to stay verticle as long as possible, so I set $\gamma=1$. I use Adam stepper with $N=30, \alpha=0.1$. Result is shown in Figure \ref{fig:cart}
	
\begin{figure}
	\centering
	\begin{subfigure}[b]{0.32\textwidth}
		\centering
		\includegraphics[width=\textwidth]{FQI_1.eps}
		\caption{Q learnt at iteration 1}
	\end{subfigure}
	\begin{subfigure}[b]{0.32\textwidth}
		\centering
		\includegraphics[width=\textwidth]{FQI_2.eps}
		\caption{Q learnt at iteration 2}
	\end{subfigure}
	\begin{subfigure}[b]{0.32\textwidth}
		\centering
		\includegraphics[width=\textwidth]{FQI_5.eps}
		\caption{Q learnt at iteration 5}
	\end{subfigure}
	\caption{Q learnt}
	\label{fig:fqi}
\end{figure}
\begin{figure}
	\centering
		\includegraphics[width=\textwidth]{J.eps}
	\caption{$J(\pi_k)$}
	\label{fig:J}
\end{figure}
\begin{figure}
	\centering
	\includegraphics[width=\textwidth]{cart.eps}
	\caption{cart-pole}
	\label{fig:cart}
\end{figure}
\end{document}
